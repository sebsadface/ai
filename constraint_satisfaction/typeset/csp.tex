%%%%%%%%%%%%%%%%%%%%% PACKAGE IMPORTS %%%%%%%%%%%%%%%%%%%%%
\documentclass{article}
\usepackage{import}
\usepackage{amsmath, amsfonts, amsthm, amssymb}
\usepackage{lmodern}
\usepackage{microtype}
\usepackage{fullpage}       
\usepackage{changepage}
\usepackage{hyperref}
\usepackage{blindtext}
\usepackage{subcaption}
\hypersetup{
    colorlinks=true,
    linkcolor=blue,
    filecolor=magenta,      
    urlcolor=blue,
    pdftitle={Overleaf Example},
    pdfpagemode=FullScreen,
    }
\urlstyle{same}

\newenvironment{level}%
{\addtolength{\itemindent}{2em}}%
{\addtolength{\itemindent}{-2em}}

\usepackage{amsmath,amsthm,amssymb}

\usepackage[nooldvoltagedirection]{circuitikz}
\usetikzlibrary{decorations,arrows,shapes}

\usepackage{datetime}
\usepackage{etoolbox}
\usepackage{enumerate}
\usepackage{enumitem}
\usepackage{listings}
\usepackage{array}
\usepackage{varwidth}
\usepackage{tcolorbox}
\usepackage{amsmath}
\usepackage{circuitikz}
\usepackage{verbatim}
\usepackage[linguistics]{forest}
\usepackage{listings}
\usepackage{xcolor}
\renewcommand{\rmdefault}{cmss}


\newcommand\doubleplus{+\kern-1.3ex+\kern0.8ex}
\newcommand\mdoubleplus{\ensuremath{\mathbin{+\mkern-10mu+}}}

\definecolor{codegreen}{rgb}{0,0.6,0}
\definecolor{codegray}{rgb}{0.5,0.5,0.5}
\definecolor{codepurple}{rgb}{0.58,0,0.82}
\definecolor{backcolour}{rgb}{0.95,0.95,0.92}

\lstdefinestyle{mystyle}{
    language=Python,
    basicstyle=\ttfamily\small,
    keywordstyle=\color{blue},
    stringstyle=\color{red},
    commentstyle=\color{green},
    morecomment=[l][\color{magenta}]{\#},
    backgroundcolor=\color{backcolour},   
    breakatwhitespace=false,         
    breaklines=true,                 
    captionpos=b,                    
    keepspaces=true,                 
    numbers=left,                    
    numbersep=5pt,                  
    showspaces=false,                
    showstringspaces=false,
    showtabs=false,                  
    tabsize=2
}

\lstset{style=mystyle}
\setlength{\parindent}{0pt}
\setlength{\parskip}{5pt plus 1pt}

\providetoggle{questionnumbers}
\settoggle{questionnumbers}{true}
\newcommand{\noquestionnumbers}{
    \settoggle{questionnumbers}{false}
}

\newcounter{questionCounter}
\newenvironment{question}[2][\arabic{questionCounter}]{%
    \ifnum\value{questionCounter}=0 \else {\newpage}\fi%
    \setcounter{partCounter}{0}%
    \vspace{.25in} \hrule \vspace{0.5em}%
    \noindent{\bf \iftoggle{questionnumbers}{Question #1: }{}#2}%
    \addtocounter{questionCounter}{1}%
    \vspace{0.8em} \hrule \vspace{.10in}%
}

\newcounter{partCounter}[questionCounter]
\renewenvironment{part}[1][\alph{partCounter}]{%
    \addtocounter{partCounter}{1}%
    \vspace{.10in}%
    \begin{indented}%
       {\bf (#1)} %
}{\end{indented}}

\def\indented#1{\list{}{}\item[]}
\let\indented=\endlist
\def\show#1{\ifdefempty{#1}{}{#1\\}}
\def\IMP{\longrightarrow}
\def\AND{\wedge}
\def\OR{\vee}
\def\BI{\leftrightarrow}
\def\DIFF{\setminus}
\def\SUB{\subseteq}


\newcolumntype{C}{>{\centering\arraybackslash}m{1.5cm}}
\renewcommand\qedsymbol{$\blacksquare$}
\newtcolorbox{answer}
{
  colback   = green!5!white,    % Background colorucyitc,
  colframe  = green!75!black,   % Outline color
  box align = center,           % Align box on text line
  varwidth upper,               % Enables multi line input
  hbox                          % Bounds box to text width
}

\newcommand{\myhwname}{CSE 473 Homework 2}
\newcommand{\myname}{Sebastian Liu}
\newcommand{\myemail}{ll57@cs.washington.edu}
\newcommand{\mysection}{AB}
\newcommand{\dollararrow}{\stackrel{\$}{\leftarrow}}
%%%%%%%%%%%%%%%%%%%%%%%%%%%%%%%%%%%%%%%%%%%%%%%%%%%%%%%%%%%

%%%%%%%%%%%%%%%%%%% Document Options %%%%%%%%%%%%%%%%%%%%%%
\noquestionnumbers
%%%%%%%%%%%%%%%%%%%%%%%%%%%%%%%%%%%%%%%%%%%%%%%%%%%%%%%%%%%

%%%%%%%%%%%%%%%%%%%%%%%% WORK BELOW %%%%%%%%%%%%%%%%%%%%%%%%
\begin{document}

\begin{center}
    \textbf{Homework 2 CSPS} \bigskip
\end{center}

%%%%%%%%%%%%%%%%%%%%%%%% Task 1 %%%%%%%%%%%%%%%%%%%%%%%%M
\begin{question}{1. Basics}
    \begin{part}[1.]
        \begin{answer}
            \begin{itemize}
                \item \textbf{Variables:} \\
                Let $B = \{B_0, B_1, \ldots, B_{23}\}$ be the set of one-hour time blocks available in a day (24 hours a day).                
                \item \textbf{Domains:}\\
                     Assume we have 4 pieces of homework, let $H = \{H_1, H_2, H_3, H_4\}$ be the set of these 4 homework tasks. Each homework piece $H_j$ has an associated amount of uninterruptible time $d_j$ hours.
                  Each time block $B_i$ has a \textbf{domain:} $D_i = \{\text{none}, H_1, H_2, H_3, H_4\}$, which represent which homework can be scheduled during one-hour block $B_i$.

                \item \textbf{Constraints:}
                \begin{itemize}
                  \item Each homework $H_j$ must be allocated to consecutive one-hour time blocks that sum up to the entire duration $d_j$.
                  \item  All homework must be completed (i.e. For each homework $H_j$, there must be a sequence of consecutive time blocks $B_{i_1}, B_{i_2}, \ldots, B_{i_{d_j}}$ such that $H_j$ is scheduled in each of these time blocks).
                  \item  No two homework can be in the same time block (i.e. there is not a time block $B_i$ such that $B_i$ is scheduled with both $H_x$ and $H_y$ where $x \neq y$).
                  \item  If $H_x$ is a prerequisite for $H_y$, then $H_x$ must be scheduled before $H_y$ (e.g. if homework $H_2$ is dependent on homework $H_1$, then $H_1$ must be finished before $H_2$).
                \end{itemize}
              \end{itemize}
        \end{answer}
    \end{part}

    \begin{part}[2.]
        \begin{answer}
            \begin{part}[a]
                Unary constraints apply to \textbf{one} variable.
            \end{part}
            \begin{part}[b]
                Binary constraints apply to \textbf{two} variables.
            \end{part}
    
            \begin{part}[c]
                Ternary constraints apply to \textbf{three} variables.
            \end{part}
    
            \begin{part}[d]
                n-ary constraints apply to \textbf{n} variables, where n is an integer greater than 0.
            \end{part}
        \end{answer}
    \end{part}
\end{question}

%%%%%%%%%%%%%%%%%%%%%%%% Task 2 %%%%%%%%%%%%%%%%%%%%%%%%
\begin{question}{2. Complex}
    \begin{part}[1.]
        \begin{answer}
            \begin{part}[a]
                \textbf{Unary:} 3 $\mid$ \textbf{Binary:} 4 $\mid$ \textbf{Ternary:} 0
            \end{part}
    
            \begin{part}[b]
                \textbf{Unary:} 1 $\mid$ \textbf{Binary:} 4 $\mid$ \textbf{Ternary:} 1
            \end{part}
        \end{answer}
    \end{part}

    \begin{part}[2.]
        \begin{answer}
            \begin{part}[a]
                \textbf{Answer:} B
            \end{part}
    
            \begin{part}[b]
                \textbf{Answer:} A
            \end{part}
    
            \begin{part}[c]
                \begin{part}[i.]
                    \textbf{Answer:} B (Constraint violated)
                \end{part}

                \begin{part}[ii.]
                    \textbf{Answer:} C (Game unsolved)
                \end{part}
            \end{part}

            \begin{part}[d]
                \textbf{Answer:} A (Unique solution)\\
                \begin{tabular}{|c|c|c|}
                    \hline
                     & 1 & 1\\
                    \hline
                    1 &  mine & \\
                    \hline
                     & 1 & 1\\
                    \hline
                \end{tabular}
            \end{part}
        \end{answer}
    \end{part}

    \begin{part}[3.]
        \begin{answer}
            \begin{part}[a]
                \textbf{Answer:} D
            \end{part}
    
            \begin{part}[b]
                \textbf{Answer:} B
            \end{part}
        \end{answer}
    \end{part}
\end{question}

%%%%%%%%%%%%%%%%%%%%%%%% Task 3 %%%%%%%%%%%%%%%%%%%%%%%%
\begin{question}{3. CSPs: Domains and Arc Consistency}
    \begin{part}[1.]
        \begin{answer}
            \textbf{For A:} 2,3,4 \\
            \textbf{For B:} 2,3,4
        \end{answer}
    \end{part}

    \begin{part}[2.]
        \begin{answer}
            \textbf{True Statements:} (i), (ii), (iii)
        \end{answer}
    \end{part}

    \begin{part}[3.]
        \begin{answer}
            \textbf{True Statements:} (i), (iii), (v)
        \end{answer}
    \end{part}
\end{question}
\end{document}
