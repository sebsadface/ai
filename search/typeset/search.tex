%%%%%%%%%%%%%%%%%%%%% PACKAGE IMPORTS %%%%%%%%%%%%%%%%%%%%%
\documentclass{article}
\usepackage{import}
\usepackage{amsmath, amsfonts, amsthm, amssymb}
\usepackage{lmodern}
\usepackage{microtype}
\usepackage{fullpage}       
\usepackage{changepage}
\usepackage{hyperref}
\usepackage{blindtext}
\usepackage{subcaption}
\hypersetup{
    colorlinks=true,
    linkcolor=blue,
    filecolor=magenta,      
    urlcolor=blue,
    pdftitle={Overleaf Example},
    pdfpagemode=FullScreen,
    }
\urlstyle{same}

\newenvironment{level}%
{\addtolength{\itemindent}{2em}}%
{\addtolength{\itemindent}{-2em}}

\usepackage{amsmath,amsthm,amssymb}

\usepackage[nooldvoltagedirection]{circuitikz}
\usetikzlibrary{decorations,arrows,shapes}

\usepackage{datetime}
\usepackage{etoolbox}
\usepackage{enumerate}
\usepackage{enumitem}
\usepackage{listings}
\usepackage{array}
\usepackage{varwidth}
\usepackage{tcolorbox}
\usepackage{amsmath}
\usepackage{circuitikz}
\usepackage{verbatim}
\usepackage[linguistics]{forest}
\usepackage{listings}
\usepackage{xcolor}
\renewcommand{\rmdefault}{cmss}


\newcommand\doubleplus{+\kern-1.3ex+\kern0.8ex}
\newcommand\mdoubleplus{\ensuremath{\mathbin{+\mkern-10mu+}}}

\definecolor{codegreen}{rgb}{0,0.6,0}
\definecolor{codegray}{rgb}{0.5,0.5,0.5}
\definecolor{codepurple}{rgb}{0.58,0,0.82}
\definecolor{backcolour}{rgb}{0.95,0.95,0.92}

\lstdefinestyle{mystyle}{
    language=Python,
    basicstyle=\ttfamily\small,
    keywordstyle=\color{blue},
    stringstyle=\color{red},
    commentstyle=\color{green},
    morecomment=[l][\color{magenta}]{\#},
    backgroundcolor=\color{backcolour},   
    breakatwhitespace=false,         
    breaklines=true,                 
    captionpos=b,                    
    keepspaces=true,                 
    numbers=left,                    
    numbersep=5pt,                  
    showspaces=false,                
    showstringspaces=false,
    showtabs=false,                  
    tabsize=2
}

\lstset{style=mystyle}
\setlength{\parindent}{0pt}
\setlength{\parskip}{5pt plus 1pt}

\providetoggle{questionnumbers}
\settoggle{questionnumbers}{true}
\newcommand{\noquestionnumbers}{
    \settoggle{questionnumbers}{false}
}

\newcounter{questionCounter}
\newenvironment{question}[2][\arabic{questionCounter}]{%
    \ifnum\value{questionCounter}=0 \else {\newpage}\fi%
    \setcounter{partCounter}{0}%
    \vspace{.25in} \hrule \vspace{0.5em}%
    \noindent{\bf \iftoggle{questionnumbers}{Question #1: }{}#2}%
    \addtocounter{questionCounter}{1}%
    \vspace{0.8em} \hrule \vspace{.10in}%
}

\newcounter{partCounter}[questionCounter]
\renewenvironment{part}[1][\alph{partCounter}]{%
    \addtocounter{partCounter}{1}%
    \vspace{.10in}%
    \begin{indented}%
       {\bf (#1)} %
}{\end{indented}}

\def\indented#1{\list{}{}\item[]}
\let\indented=\endlist
\def\show#1{\ifdefempty{#1}{}{#1\\}}
\def\IMP{\longrightarrow}
\def\AND{\wedge}
\def\OR{\vee}
\def\BI{\leftrightarrow}
\def\DIFF{\setminus}
\def\SUB{\subseteq}


\newcolumntype{C}{>{\centering\arraybackslash}m{1.5cm}}
\renewcommand\qedsymbol{$\blacksquare$}
\newtcolorbox{answer}
{
  colback   = green!5!white,    % Background colorucyitc,
  colframe  = green!75!black,   % Outline color
  box align = center,           % Align box on text line
  varwidth upper,               % Enables multi line input
  hbox                          % Bounds box to text width
}

\newcommand{\myhwname}{CSE 473 Homework 1}
\newcommand{\myname}{Sebastian Liu}
\newcommand{\myemail}{ll57@cs.washington.edu}
\newcommand{\mysection}{AB}
\newcommand{\dollararrow}{\stackrel{\$}{\leftarrow}}
%%%%%%%%%%%%%%%%%%%%%%%%%%%%%%%%%%%%%%%%%%%%%%%%%%%%%%%%%%%

%%%%%%%%%%%%%%%%%%% Document Options %%%%%%%%%%%%%%%%%%%%%%
\noquestionnumbers
%%%%%%%%%%%%%%%%%%%%%%%%%%%%%%%%%%%%%%%%%%%%%%%%%%%%%%%%%%%

%%%%%%%%%%%%%%%%%%%%%%%% WORK BELOW %%%%%%%%%%%%%%%%%%%%%%%%
\begin{document}

\begin{center}
    \textbf{Homework 1} \bigskip
\end{center}

%%%%%%%%%%%%%%%%%%%%%%%% Task 1 %%%%%%%%%%%%%%%%%%%%%%%%M
\begin{question}{1. State Space}
    \begin{part}[1.]
        \begin{answer}
            \textbf{Minimal state representation:}\\
            Since the bugs do not have fixed starting positions and the goal is for them to end up on the same square,
            we can represent the state by the relative positions of the bugs w.r.t. each other.
            We could express this relative distance as \((dx, dy)\), where \(dx\) and \(dy\) represent the difference in x and y coordinates of the bugs.
            Since we have a map of size \(M \times N\), we have $dx \in [-M +1, M-1]$ and $dy \in [-N + 1, N -1 ]$.\\
            The state representation of $(dx, dy)$ give us the following information:\\
              - The bugs are on the same square if \(dx = 0\) and \(dy = 0\).\\
              - The bugs' direction relative to each other.\\
              - The bugs' distance relative to each other.\\
            When either or both of the bugs move, the state $(dx, dy)$ will change, and we can use this information to determine the next possible states.
            
        \end{answer}
    \end{part}

    \begin{part}[2.]
        \begin{answer}
            \textbf{Size of the state space:}\\
            Given the size of the maze is \(M \times N\), the max and min values for \(dx\) are \([-M+1, M-1]\) and for \(dy\) are \([-N+1, N-1]\).\\
            Since we both $dx$ and $dy$ can be of value $0$, we have the state space size:
            \[ (M -1 - (-M + 1) + 1) \times (N - 1 - (-N + 1) + 1) =  (2M - 1) \times (2N - 1)\]
        \end{answer}
    \end{part}
\end{question}

%%%%%%%%%%%%%%%%%%%%%%%% Task 2 %%%%%%%%%%%%%%%%%%%%%%%%
\begin{question}{2. General Search}
    \begin{part}[1.]
        \begin{answer}
            $S-G$
        \end{answer}
    \end{part}

    \begin{part}[2.]
        \begin{answer}
            $S-A-C-G$
        \end{answer}
    \end{part}

    \begin{part}[3.]
        \begin{answer}
            $S-A-B-D-G$
        \end{answer}
    \end{part}

    \begin{part}[4.]
        \begin{answer}
            $S-A-C-G$
        \end{answer}
    \end{part}

    \begin{part}[5.]
        \begin{answer}
            Base on the graph, we know $h^*(S) = 7$, $h^*(A) = 6$, $h^*(B) = 6$, $h^*(C) = 3$, $h^*(D) = 3$, $h^*(G) = 0$.\\
        \begin{part}[a]
            Yes. It never overestimates the true cost to reach the goal from any node ($h_1(S)= 4 \le h^*(S) = 7$, $h_1(A) = 2 \le h^*(A) = 6$, $h_1(B) = 5 \le h^*(B) = 6$, $h_1(C) = 2 \le h^*(C) = 3$, $h_1(D) = 3 \le h^*(D) = 3$, $h_1(G) = 0 \le h^*(G) = 0$).
        \end{part}

        \begin{part}[b]
            Yes. $h_1(S) - h_1(A) = 2 \le 1 = c(S,A) $, $h_1(S) - h_1(G) = 4 \le 7 = c(S,G)$, $h_1(A) - h_1(B) = -3 \le 2 = c(A,B)$, $h_1(A) - h_1(C) = 0 \le 3 = c(A,C)$, $h_1(B) - h_1(D) = 2 \le 3 =c(B,D) $, $h_1(C) - h_1(G) = 2 \le 3 = c(C,G)$, $h_1(C) - h_1(D) = -1 \le 6 = c(C,D)$, $h_1(D) - h_1(G) = 3 \le 3 = c(D,G)$.

        \end{part}

        \begin{part}[c]
            Yes. Similar to $h_1$, it never overestimates the true cost ($h_2(S) = 3 \le h^*(S) = 7$, $h_2(A) = 2 \le h^*(A) = 6$, $h_2(B) = 6 \le h^*(B) = 6$, $h_2(C) = 1 \le h^*(C) = 3$, $h_2(D) = 3 \le h^*(D) = 3$, $h_2(G) = 0 \le h^*(G) = 0$).

        \end{part}

        \begin{part}[d]
            Yes. $h_2(S) - h_2(A) = 1 \le 1 = c(S,A) $, $h_2(S) - h_2(G) = 3 \le 7 = c(S,G)$, $h_2(A) - h_2(B) = -4 \le 2 = c(A,B)$, $h_2(A) - h_2(C) = 1 \le 3 = c(A,C)$, $h_2(B) - h_2(D) = 3 \le 3 =c(B,D) $, $h_2(C) - h_2(G) = 1 \le 3 = c(C,G)$, $h_2(C) - h_2(D) = -2 \le 6 = c(C,D)$, $h_2(D) - h_2(G) = 3 \le 3 = c(D,G)$.

        \end{part}

        \begin{part}[e]
            Yes. The max of two admissible heuristics is still admissible as it is guaranteed to never overestimate the cost.
        \end{part}
    \end{answer}
    \end{part}
\end{question}

%%%%%%%%%%%%%%%%%%%%%%%% Task 3 %%%%%%%%%%%%%%%%%%%%%%%%
\begin{question}{3. Multi-agent Search}
    \begin{part}[1.]
        \begin{answer}
            \begin{part}[a]\\
                i. Node P: Avg. of payoffs considering each bid with equal probability: \(\frac{0 + 3 + 0 + 1 + 0}{5} = 0.8\)

                ii. Node Q: \(\frac{1 - 2 + 1 + 2 + 0}{5} = 0.4\)

                iii. Node R: \(\frac{1 + 2 - 3 + 2 + 0}{5} = 0.4\)

                iv. Node S:  \(\max(P, Q, R) \max(0.8, 0.4, 0.4) = 0.8\). Node S = 0.8.
            \end{part}
    
            \begin{part}[b]
                Given the values of \(P, Q, R\) from (a), Alyssa should bid \(x_1\) since it leads to the highest payoff.
            \end{part}
        \end{answer}
    \end{part}
\end{question}

%%%%%%%%%%%%%%%%%%%%%%%% Task 4 %%%%%%%%%%%%%%%%%%%%%%%%
\begin{question}{4. Alpha-Beta}
    \begin{part}[1.]
        \begin{answer}
            possible

        \end{answer}
    \end{part}

    \begin{part}[2.]
        \begin{answer}
            not possible
        \end{answer}
    \end{part}
\end{question}
\end{document}
